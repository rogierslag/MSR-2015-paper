\documentclass[conference]{IEEEtran}

\usepackage{graphicx}
 \usepackage{amsmath}

\usepackage{xspace} 
\usepackage{hyperref}
% cleverref has to be loaded after hyperref
\usepackage[noabbrev]{cleveref}
\usepackage{color}
\usepackage{amssymb,amsmath}
\usepackage{multirow}
\usepackage{tcolorbox}

\renewcommand{\quotation}[1]{``\emph{#1}''}

\newcommand{\ie}{\emph{i.e.},\xspace}
\newcommand{\eg}{\emph{e.g.},\xspace}
\newcommand{\etc}{\emph{etc.}\xspace}
\newcommand{\etal}{\emph{et al.}\xspace}

%\usepackage[none]{hyphenat}

\usepackage{paralist}

\newcommand{\nb}[3]{
  \fcolorbox{black}{#2}{\bfseries\sffamily\scriptsize#1}
    {\sf\small$\blacktriangleright$\textit{#3}$\blacktriangleleft$}
}

\newcommand\AB[1]{\nb{alberto}{cyan}{#1}}

\newcommand\odf{\emph{one-day flies}\xspace}
\newcommand\Odf{\emph{One-day flies}\xspace}
\newcommand\ru{regular users\xspace}


\tcbset{center title,center upper,fonttitle=\bfseries}

\hyphenation{op-tical net-works semi-conduc-tor}


\begin{document}
\title{One-day flies on StackOverflow\\
{\LARGE Why the vast majority of StackOverflow users only posts once}}



\author{
\IEEEauthorblockN{
Rogier Slag,\IEEEauthorrefmark{1}
Mike de Waard,\IEEEauthorrefmark{2}
Alberto Bacchelli\IEEEauthorrefmark{3}}
\IEEEauthorblockA{
 \{\IEEEauthorrefmark{1}r.g.j.slag,
   \IEEEauthorrefmark{2}m.dewaard\}@student.tudelft.nl,
 \IEEEauthorrefmark{3}a.bacchelli@tudelft.nl}
\IEEEauthorblockA{Delft University of Technology, The Netherlands}}





\maketitle


\begin{abstract}


StackOverflow (SO) is a popular question and answers (Q\&A) platform related to
software development. An interesting characteristic of SO is that about 
half of its users makes only one contribution to the platform in total.

In this work, we study this group of users, which we call \odf, and we 
investigate why they do not continue to contribute to the platform.
By achieving this understanding we can find ways to enable
users to become more active. 
\end{abstract}

\IEEEpeerreviewmaketitle



\section{Introduction}

StackOverflow is a Q\&A platform and community on the Internet that focuses
primarily on programming related questions. SO has been gaining huge popularity
during the last few years \cite{anderson2012discovering} and the availability
of its data dump to the public has triggered numerous investigations (\eg
\cite{treude2011programmers,barua2014developers,morrison2013age}).  Many
studies on SO cover aspects related to users, such as their personalities
\cite{bosu2013building}, their field of expertise, and on how they contribute
\cite{movshovitz2013analysis}. Little is know, yet, about the underlying
structure of the StackOverflow community, in particular the distribution of
posts and reputation among users.

In this paper, we try to start filling this gap by analyzing less active users,
\ie those who post only once during their whole registration period.  To this
aim, we first quantify these users finding that they account for about half of
the users of SO. Then, we use different information sources to try to
quantitatively characterize this set of users, which we call \odf, by
contrasting them with more active participants.  Finally, we conduct an initial
qualitative investigation to better understand what could induce these users to
stop contributing.  Results show that many of our hypotheses of why users
become \odf do not fully explain this phenomenon. We discuss aspects emerged as 
candidates for future evaluation.



\section{Related work}

Several aspects of SO have already been investigated, such as how to build
reputation \cite{bosu2013building} and how the reputation system encourages
users to participate \cite{movshovitz2013analysis}. These studies gave us
insight in how the reputation system works and how this affects the user base. 
Correa \& Sureka \cite{correa2014chaff} investigated the deletion of
questions on SO; we use this as the basis for verifying some of our hypotheses.
Bazelli \etal analyzed personality traits of SO
users~\cite{bazelli2013personality}; on their work we base our sentimental
analysis for post and answers. 
Yang \etal investigated the group of experts on
SO~\cite{yang2014sparrows}; our work complements theirs by considering a
different group of users.


\section{Defining one-day flies}

As shown by several studies~\cite{yang2014sparrows}, Q\&A platforms are
composed by users with different activity behaviors, which closely follow a
power law distribution: A small set of very highly active users contributes
to the vast majority of the produced content; a larger set of less active users
contributes to fewer parts of the content; and a set (the largest) of users
contributes very little to nothing.

We start our investigation by quantifying the result of this community dynamics
in SO and use it to cluster users and define \odf. We consider the SO data dump
provided for the MSR Challenge 2015~\cite{MSRChallenge2015} and we analyze both
user activity (based on number of posts) and user reputation (based on the
score obtained in the gamification mechanism in place on
SO~\cite{anderson2013steering}).  We find that 47\% of the users made a maximum
of one post, while the most prolific user had 30.043 posts. Similarly, 60\%
(2.063.174) of the users had a reputation of one, the third quartile is 13, and
the maximum value is 709.269.  This is in line with general trends on Q\&A
platforms.

We then apply clustering to users based on their reputation, using k-means.
The largest group consists of 3.469.897 users, who have between 0 and 1 of
reputation and posted questions; the medium active user group is significantly
smaller (3.012 users) with an average reputation of 38.000;  the group with the
most reputation is by far the smallest (186 users) with a very high average
reputation of ca. 200.000.

With this information, we define \odf as users who posted a maximum of one
question in SO, thus covering both low activity and low reputation. We only
consider users that are registered for more than 6 months to avoid that
recently registered users could bias the results. The resulting set comprises
1.622.688 users, \ie 47\% of the user base. 



\section{Experiment}

We conduct an experiment to quantitatively test different hypotheses, expressed
as research questions on what may characterize \odf. In the experiment we use
the complete SO MSR Challenge dump~\cite{MSRChallenge2015}, imported into a
MSSQL database. For the actual querying, statistical analysis, and plotting of
the data we use a mixture of Ruby (especially to mine additional data), C\# (to
query the data), and R (to perform statistical analysis and generate plots).
In the following we describe each hypothesis, the intuition behind it, the
research method we used to test it, and the results.\\

\begin{tcolorbox}[size=fbox,title=RQ1: Duplicate Questions]
Are \odf more likely to create questions eventually marked as duplicates?
\end{tcolorbox}

\textbf{Intuition.} New users who register on
the website could, because of their unfamiliarity with the system, ask
questions that have been asked previously. These questions would be flagged
by the community as \textit{duplicate}. Although this may be the appropriate
action for the community as a whole, it could be extremely discouraging for the
user which posted the question, thus inducing to not post anymore. 

\textbf{Research method.} To answer this research question, we compare the
percentage of duplicate questions  between and \odf and the other users
(defined from now on as \emph{\ru}). To do so, we
extract questions asked by \odf and count those marked as duplicate by checking
on  LinkTypeId in the PostLinks table; then we do the same for questions asked
by \ru. After retrieving these results we compare them.

\textbf{Results.} The total amount of questions asked by \odf is about 8\%. The
number of duplicate questions is surprisingly low for both groups. One could
expect that this would be high on a public Q\&A community where a large
percentage of the users is unfamiliar with platform. However, only $2.2\%$ of
the one-day flies questions is marked as a duplicate. Surprisingly, this
percentage increases to $2.9\%$ for \ru; we expected \ru to be more familiar
with the platform and its search functionality. Based on these results, \odf
are not more likely to create duplicate questions than \ru, but
slightly less.\\


\begin{tcolorbox}[size=fbox,title=RQ2: Uncommon Tags]
Are \odf more likely to post questions with uncommon tags?
\end{tcolorbox}

\textbf{Intuition.} The topic of a question in SO is commonly translated into
tags, which make it easier for experts to find questions they could answer.
Users who post questions regarding a topic with hardly any experts on the SO
platform can get discouraged by the lack of comments or answers to guide them
into the right direction. This types of topics would be reflected by uncommon
tags accompanying the questions. 
 
\textbf{Research method.} Correa and Sureka found over $3.300$ tags that only
occur in questions of quality far below StackOverflow
standards~\cite{correa2014chaff}. The answer to our research question is
inspired by their findings, as similar results might be found. We analyze the
top $50$ most popular tags given by \odf questions in comparison to the top
$50$ tags given in all questions in SO. This to get insights on the
most popular tags and any popular uncommon topics among \odf.

\textbf{Results.} Surprisingly the top-5 tags of both \odf and \ru
are identical (although ordered different). Although there are a few tags that
are specific to \odf, most of them are very similar. Based on these results, the
answer to our research question is negative.

\begin{tcolorbox}[size=fbox,title=RQ3: Deleted Questions]
Are \odf more likely to have their posts removed (either by
themselves or by a moderator)?
\end{tcolorbox}

\textbf{Intuition.} Another possible explanation for the large number of users
with a low question count can also be found in deletions. Users themselves can
delete questions after posting them, and so can the moderators in the
community. This can have quite an impact. As studied in \cite{correa2014chaff},
deleted questions usually are of a very low quality (the quality is considered
low compared to the expectation of the SO community). Users unfamiliar with the
ethos of StackOverflow may inadvertently post such questions, only later
realizing these do not belong there, and be deleted (or be deleted by
moderators). 

\textbf{Research method.} The number of deletions is not directly obtainable
from the database dump as provided by StackOverflow: The only way to know
whether a question on the data dump has been eventually removed is to check its
online existence on the SO website. Due to our resources we could not check
each question in the dump against the website, so we used random sampling. We
select two samples of 150.000 questions: One for \odf and one for \ru.
For each of these sets, we got the SO page and parsed its DOM structure. This
allowed us to extract whether the question was deleted, or possibly closed.

\textbf{Results.} The number of deleted questions for \odf was $15.4\%$,
whereas it was $10.9\%$ for the high post count users. However, it cannot
easily be determined what the reason for deletion was. As explained in
\cite{correa2014chaff}, there are several reasons for a question to be deleted.
Therefore further investigation would be
necessary to give a more complete answer to this question.  The same analysis
also gave interesting insights in the number of closed posts.  For both groups,
the ratio of closed posts is surprisingly low: $0.92\%$ for \odf and $0.76\%$
for high post count users. Questions can be closed for various reasons, such as
being a duplicate or for being off-topic. Closed questions may be improved or
deleted later on. The difference is low, thus we can give a 
negative answer to the research question when considering closed posts.\\ 


\begin{tcolorbox}[size=fbox,title=RQ4: View Count]
Are \odf more likely to attract less views? 
\end{tcolorbox}

\textbf{Intuition.} Users' questions may be not
viewed as much as they had anticipated, resulting in fewer good answers. It may
be the case their question was listed lower due to their lack of reputation, or
simply got lost in the list of questions. It is also possible
that questions from people with no post history simply get opened less often,
resulting in lower quality answers. A low view count can discourage the 
question askers, since they may feel neglected by the community. 
 
\textbf{Research method.} To answer our research question, we distinguish
between \odf and all users. To examine the results,
we eliminate the highest viewed questions and the least
viewed questions  from the data set: We consider different cut-offs
(\ie 5\%, 10\%, and 20\%) and compute separate results. 
 
\textbf{Results.} Both groups have an average view count of ca. $190$ ($198$
for \odf, and $187$ for regular users), but the standard deviation is high
($400$ for \odf, over $1.000$ for all). To compensate for these
effects, the dataset was modified to only include the posts which fitted into
the $10\%$ to $90\%$ range of the view counts and the results were rerun.
Using this reduced data set, with the extremes removed, we obtain a similar
result, with an average view count of $127$ for \odf and $113$ for all. The standard deviations dropped a lot as well, to only $100$
respectively $93$. When the analysis is repeated by eliminating the extreme
$5\%$ of $20\%$, similar results remain. Therefore we conclude that the posts
of \odf are actually not viewed less.\\


\begin{tcolorbox}[size=fbox,title=RQ5: Unanswered Questions]
Are \odf more likely to receive no answer to their questions?
\end{tcolorbox}

\textbf{Intuition.} New users simply got let down by the community: They asked
a question which did not violate any of the StackOverflow rules, but did not
receive a (satisfactory) answer anyway. This would probably the most
discouraging situation, since the user did not receive help due to aspects out
of his or her control. That might cause them to lose the confidence in the
community itself. 
 
\textbf{Research method.} To answer our question, we compared the set of
questions done by \odf to the set of questions done by other users.

\textbf{Results.} Based on the data, we found the first non-negligible
difference between the groups. \odf have a larger percentage of unanswered
questions (\ie $17\%$) compared to the other group (\ie $10\%$). It could be
that our intuition is correct and part of the \odf leave SO because they get
demotivated by their questions left unanswered. Especially if they saw other
users receive answers to their posts much more often. Since this difference 
is only 7\% we can answer our research question
positively, but it is not sufficient to account for the vast number of \odf.



\section{In-depth investigation}\label{QualitativeResearch}

Since the results did not reveal a clear reason that could completely account
for the whole large number of \odf, we decided upon a qualitative 
research to get new ideas and see if something was overlooked that can be found
in the data. To this aim, we manually analyzed some posts and their
accepted answer. We found the following parameters which could influence the
motivation: answer niceness, and the use of code in the question or answer.

The investigation on niceness might be interesting. There still were some answers
which could not be considered to be very nice, but were still accepted as answer.
To investigate this deeper, we did a completer analysis on a random sample of
$10000$ questions and their accepted answers using SentiStrength \cite{thelwall2010sentiment}.
The result was a number on a scale of $-4$ (very negative) to $4$ (very positive).
This analysis was run for both the questions and accepted answers for the questions
asked by \odf, as well as for questions asked by \ru.

This yielded an average niceness of the \odf questions of $0.196$. The accepted
answers for this segment had a rating of $0.09$. For \ru these numbers respectively
were $0.080$ and $0.059$. These numbers suggest that \odf get "nicer" answers
compared to regular users.



Codewise, a search on the data set yielded that for \odf, 70\% of their questions contained
a code block. For \ru, this was 65\%. Although this is done on the entire data set,
it does not yield information on the quality of the code in the questions. This
may be a topic for future research.
The same paradigm is repeated for the answers. Here we see that 65\% of the \odf
accepted answers contained code, compared to 58\% for the \ru.

\section{Discussion} \label{NewResearchQuestions}

%\Cref{fig:finalresults} presents a summary of the results obtained by
%analyzing statistically the features of \odf in contrast to regular users.



The presented results are not conclusive about what could motivate the behavior
of \odf. As indicated by the previous sections, there does not seem to be a clear reason
within these statistics why \odf contribute less to the platform.  However, our
results gave insights on some aspects that might be interesting venues for
future work, we describe some of those in the following. As these factors have not
een investigated in-depth, they serve as future options for exploration.

It is worth noting that the analyses also have been performed with other characteristics
for the definition of a \odf. However, this did not yield different results, which is why the
numbers have been ommitted.

%Assumption of bad code examples
\subsection{Code example quality}

In general it seems the quality of the code examples in questions of
\odf is lower compared to the quality of users with a higher
reputation. \Odf generally include too much code (by simply pasting
all the code into the questions code block). This discourages users to form an
adequate explanation.

Apart from that, the general code quality itself is lower as well. Code is
poorly modularized, which causes people to respond to those issues, before
addressing the original problem of the question asker. He or she may then get
irritated, since they receive feedback they simply are not ready for at that
point in the learning process. The same goes for the people trying to figure
out what the issue is: They need to wrap their mind around to some non-standard
principles used by others, which are (from their perspective) less easy to
reason about.

%Assumption of negative feedback from the community
\subsection{Negative feedback}

As stated in the previous paragraph, it is often the case the StackOverflow
community gives tips and advice to novices how certain problems should be
structured. However, for the question asker this can be seen as negative
feedback. It does not solve their problem, while it increases the work they
have to do to solve the problem or even get community support.

Even without this feedback, the StackOverflow community has strict rules on the
type of questions allowed on the platform. A quick search over time showed the
authors that these rules were not strictly followed in the early days of the
StackExchange platform, but are more followed these days. This causes many
beginners questions, subjective discussions (\textit{Is A better than B?}), or
too-localized questions (\textit{What does this regular expression do?}) to be
voted for closing or deletion.

Since the community has answered a large number of questions, one can expect
the same thing to happen for other things, such as \textit{marked as
duplicate}. The chances are increasing that somewhere in the database, a very
similar question might have been asked already, and received an appropriate
answer. Meanwhile, a beginner on the platform might not be aware of the exact
terminology to find that specific question. Therefore they may consider their
problem to be new and unique, causing them to post it. Several minutes later,
they may find their question marked as a duplicate combined with a snide
comment on ``how you can search on StackOverflow.'' It is no wonder that a user
could see this as a negative feedback from the community.


%This section is written after writing the qualitative research section thus
%doesn't have to be revised (only spell/grammar checked)
\subsection{Self-answering}
In section \ref{QualitativeResearch}, we found that some of the
questions accepted answers were given by the askers themselves. This
could be done for documentation purposes, but also for attempting to gain
reputation, while other answers might be better. However in order to explore
this further, it is first interesting to evaluate whether \odf answer their 
own questions more often than \ru.

A user does not get reputation for answering their own question (but they can
get reputation if the answer is upvoted by others). Here there seems to be a
clear trend: users have found and answer, and decided to share it with the
community (even though they do not have a direct profit from it themselves).
This seems to be an indication these users are not scared away from
StackOverflow at that point (otherwise they likely would not have posted that
answer). 

\subsection{Everything can be found}
During the research we had several discussions with software developers
regarding this high amount of \odf. In most of these sessions
developers said to never ask a question on StackOverflow because the answer to
their question was already available. This gave the authors a new idea that
possibly almost all questions can be found, thus the chance that a user has two
questions that are not answered on StackOverflow already is rather slim. This
could cause the community to look like it has many \odf whereas the
\odf have only one question that could not be found on StackOverflow
before. If this is the case, then when taking into account the user growth, one
should see a decrease in the amount of questions asked over time. 

\section{Conclusion}

In this paper we analyzed an aspect of the underlying structure of the
SO community: the behavior of \odf. These are users who post only one question
during their participation to SO, even if they have been registered for six 
months at least. 

We investigate five research questions, based on different information sources,
to verify intuitions we have on the behavior of \odf. Results show that \odf
are not more likely to create duplicate questions, to use uncommon tags, or to
receive less views for their questions. But their questions do get removed
slightly more frequently, either by themselves or a moderator, and they are
more likely (7\%) to receive no answer to their question. Although these two
facts could explain part of the reason why \odf participate less, they do not
seem to account for the whole story. More research is necessary to understand
this phenomenon better.

\bibliographystyle{abbrv}
\bibliography{onedayflies}


% that's all folks
\end{document}
